% vim: expandtab tabstop=2 softtabstop=2 shiftwidth=2
\documentclass[a4paper,12pt]{article}
%\usepackage[scale=.8]{geometry}
\usepackage[english,noconfigs]{babel}
\usepackage{ifluatex,ifxetex}
\ifnum 0\ifxetex 1\fi\ifluatex 1\fi=0 % if pdftex
  \usepackage[utf8]{inputenc}
  \usepackage[T1]{fontenc}
\fi

\usepackage[ruled]{algorithm2e}
\usepackage[toc,page]{appendix}
\usepackage{minted}

\usepackage{graphicx}
\graphicspath{{graphics/}}

\usepackage{lipsum}

\title{%
  Universal predictor\\\small%
  Pattern-based prediction algorithms,\\%
  applications \& variations%
}

\author{%
  Carolina De Senne Garcia\\%
  Clément Durand%
}

\begin{document}
\maketitle

\vspace*{\fill}

\begin{abstract}
  \lipsum[1-2]
\end{abstract}

\vspace*{\fill}

\clearpage

\tableofcontents

\clearpage

\section*{Introduction}
\addcontentsline{toc}{section}{Introduction}

In this work we considered the problem of implementing an universal predictor based on patter matching. An universal predictor aims to perform well without knowing knowing the underlying probabilistic model of the data to be analysed.

In section \ref{paper_algos} we present an implementation of the basic idea of a predictor proposed by Ehrenfeucht and Mycielski \cite{basic_algo}, called \textit{Sampled Pattern Matching} (SPM) in \cite{paper}. Though the former was not an universal predictor, it was a good density estimator and was used by Jacquet et al. \cite{paper} as the base idea to develop an universal predictor. The implementation of this universal predictor is also presented in \ref{paper_algos}.

Section \ref{variation_algo} exposes a variation of the predictor algorithm proposed by us and based on Machine Learning techniques and pre-computation. We compare its performance with the universal predictor of section \ref{paper_algos}.

Finally, in section \ref{tests} we discuss performance tests for the algorithms implemented in this work, as well as implementation issues and possible solutions for them.

\section{Universal prediction algorithm}\label{paper_algos}

  In this section we present the algorithms from \cite{paper} which were implemented and tested in our work: the simplified version uses the longest suffix pattern match in the text we want to predict. The second version adds a parameter $\alpha$ to check for pattern matches of size $\alpha$ times the longest match size. The algorithms details are explained in their corresponding subsections bellow.

  \subsection{Simplified version}

  The base algorithm consists in finding the longest substrings from the input stream that match with the stream suffix (and that are not the suffix itself, of course). It then takes the most recent substring (in terms of the stream input data) from the set of longest substrings (which can be of size 1) and returns the character that follows this substring in the input data as a prediction. These steps are described in algorithm \ref{simplified}.

    % Explain the idea of the algorithm, and mention the algorithm

    \begin{algorithm}
      \KwData{Character flow \textit{input}.}
      \KwResult{Prediction \textit{output}.}

      \textit{patterns} \gets{}
        $\left\{ (\textit{substring},x) :%
        \textit{substring}\cdot x\subset\textit{input} \wedge
        \textit{input} = w\cdot\textit{substring} \right\}$
        with multiplicity\;

      \textit{maxlen} \gets{} $\max\{\#\textit{pattern} :
        (\textit{pattern},x)\in\textit{patterns}\}$\;

      \KwRet{$x$} where $(\textit{pattern},x)\in\textit{patterns}$ is the most
        recent pattern of length \textit{maxlen}\;

      \caption{\label{simplified}Simplified version of universal prediction.}
    \end{algorithm}

    The implementation for this algorithm is exposed in annex \ref{code_simplified}.

    % First implementation (annex A), non-optimized pattern search.

  \subsection{Adding refinements}

  The algorithm \ref{complete} introduced in \cite{paper} finds the longest suffix match, as previously. However, it includes a parameter $\alpha$ that is multiplied by the length of the longest suffix match found, giving a new length. We look at the suffix matches that have this new lenght and return the prediction character (the character following the match) with higher frequency among this set.

    \begin{algorithm}
      \KwData{Character flow \textit{input}, parameter $\alpha$.}
      \KwResult{Prediction \textit{output}.}

      \textit{patterns} \gets{}
        $\left\{ (\textit{substring},x) :%
        \textit{substring}\cdot x\subset\textit{input} \wedge
        \textit{input} = w\cdot\textit{substring} \right\}$
        with multiplicity\;

      \textit{len} \gets{} $\alpha\times\max\{\#\textit{pattern} :
        (\textit{pattern},x)\in\textit{patterns}\}$\;

      \KwRet{$x$} the most frequent among the patterns $(\textit{pattern},x)$
        of length \textit{len} in \textit{patterns}\;

      \caption{\label{complete}Universal prediction.}
    \end{algorithm}

    The implementation for this algorithm is exposed in annex \ref{code_complete}.

    \section{Variation proposal}\label{variation_algo}

    In certain contexts (e.g. languages) it could be useful to have an algorithm that learns from existing text  before performing a prediction. In this section we propose an algorithm that does so.
  
  \subsection{Learning algorithm}

  \begin{algorithm}
    \KwData{Texts to read \textit{input}, memory size \textit{cache}.}
    \KwResult{Substrings and number of occurrences.}

    \textit{substrings} \gets{}
      $\left\{ \textit{substring} : \textit{substring}\subset\textit{input}
      \wedge \#\textit{substring} \leq \textit{cache} \right\}$
      with multiplicity\;

    \KwRet{\textit{substrings}} with multiplicity count\;

    \caption{\label{learning}Learning patterns by reading.}
  \end{algorithm}

  \subsection{Prediction algorithm}

  \begin{algorithm}
    \KwData{Substrings and multiplicity counts.}
    \KwResult{Predictions}



    \KwRet{\textit{substrings}} with multiplicity count\;

    \caption{\label{predicting}Turning substrings into predictions}
  \end{algorithm}

  \subsection{Comparison}

  % Performance: difficilement comparable car on change l'équilibre du compromis
  % précomputation/prédiction.

  \section{Performance tryouts and issues}\label{tests}

  \subsection{Increasing the data sizes}

  \subsection{Distributing the algorithms}

\section*{Lessons learned}
\addcontentsline{toc}{section}{Lessons learned}

  \lipsum[8-9]

\clearpage
\begin{appendices}

  The algorithms presented in this report were implemented and evaluated
  in \emph{python3.5}. The code snippets below are simplified versions of
  the relevant parts of the implementation and are not $100\%$ correct for
  simplicity sake. Comments were added for clarity and the full source code
  remains available.

  \section{Simplified universal prediction}\label{code_simplified}

    \inputminted[linenos]{python}{code/simplified.py}

  \clearpage
  \section{Universal prediction}\label{code_complete}

    \inputminted[linenos]{python}{code/complete.py}

  \clearpage
  \section{Learning and predicting}

    \subsection{Learning algorithm}

      \inputminted[linenos]{python}{code/learning.py}

    \clearpage
    \subsection{From substrings to predictions}

      \inputminted[linenos]{python}{code/predicting.py}

\end{appendices}

\bibliographystyle{plain}
\bibliography{references} 

\end{document}
