% vim: expandtab tabstop=2 softtabstop=2 shiftwidth=2
\documentclass[a4paper,12pt]{article}
%\usepackage[scale=.8]{geometry}
\usepackage[english,noconfigs]{babel}
\usepackage{ifluatex,ifxetex}
\ifnum 0\ifxetex 1\fi\ifluatex 1\fi=0 % if pdftex
  \usepackage[utf8]{inputenc}
  \usepackage[T1]{fontenc}
\fi

\usepackage[ruled]{algorithm2e}
\usepackage[toc,page]{appendix}
\usepackage{minted}

\usepackage{graphicx}
\graphicspath{{graphics/}}

\usepackage{lipsum}

\title{%
  Universal predictor\\\small%
  Pattern-based prediction algorithms,\\%
  applications \& variations%
}

\author{%
  Carolina De Senne Garcia\\%
  Clément Durand%
}

\begin{document}
\maketitle

\vspace*{\fill}

\begin{abstract}
  \lipsum[1-2]
\end{abstract}

\vspace*{\fill}

\clearpage

\tableofcontents

\clearpage

\section*{Introduction}
\addcontentsline{toc}{section}{Introduction}

\section{Universal prediction algorithm}

  \subsection{Simplified version}

  \begin{algorithm}
    \KwData{Character flow \textit{input}.}
    \KwResult{Predictions \textit{output} and success rate \textit{success}.}

    \caption{\label{simplified}Simplified version of universal prediction.}
  \end{algorithm}

  \subsection{Adding refinements}

\section{Variation proposal}

  % Blabla: dans certains contextes (langues fixées, etc.) il peut être censé
  % d'avoir un algorithme qui apprend avant de faire de la prédiction.
  % Proposition d'un tel algorithme.

  \subsection{Learning algorithm}

  \subsection{Prediction algorithm}

  \subsection{Comparison}

  % Performance: difficilement comparable car on change l'équilibre du compromis
  % précomputation/prédiction.

\section{Performance tryouts and issues}

  \subsection{Increasing the data sizes}

  \subsection{Distributing the algorithms}

\section*{Lessons learned}
\addcontentsline{toc}{section}{Lessons learned}

  \lipsum[8-9]

\clearpage
\begin{appendices}
  \section{Simplified universal prediction}

    \inputminted[linenos]{python}{code/simplified.py}

  \clearpage
  \section{Universal prediction}

    \lipsum[11-12]

  \clearpage
  \section{Learning and predicting}

    \subsection{Learning algorithm}

      \inputminted[linenos]{python}{code/learning.py}

    \clearpage
    \subsection{From substrings to predictions}

      \inputminted[linenos]{python}{code/predicting.py}

\end{appendices}

\end{document}
