% vim: expandtab tabstop=2 softtabstop=2 shiftwidth=2
\documentclass[aspectratio=169]{beamer}
\usepackage[english,noconfigs]{babel}
\usepackage{ifluatex,ifxetex}
\ifnum 0\ifxetex 1\fi\ifluatex 1\fi=0 % if pdftex
  \usepackage[utf8]{inputenc}
  \usepackage[T1]{fontenc}
\fi

\usepackage{beamertpt}
\graphicspath{{graphics/}}

\usepackage{tikz}
\usetikzlibrary{trees}

\setbeamercovered{transparent=15}

\title[Pattern-based prediction algorithms, applications and variations]
      {Universal predictor}
\author{%
  Carolina De Senne Garcia\\%
  Clément Durand%
}

\renewcommand{\arraystretch}{1}
\setlength{\tabcolsep}{0.5ex}
\setlength{\fboxsep}{0.4ex}

\colorlet{fboxcolor}{white}
\newcommand\cfbox[1]{%
  \colorlet{currentcolor}{.}%
  {\color{fboxcolor}\fbox{\color{currentcolor}#1}}%
}

\begin{document}
\maketitle

\begin{frame}{Roadmap}\Large
  \begin{itemize}
    \item Algorithms
      \begin{itemize}
        \item Simplified predictor
        \item Universal predictor
        \item Variation proposal
      \end{itemize}
    \item Experimentation \& comparison
  \end{itemize}
\end{frame}

\tptsection{Simple predictor}

\begin{frame}{Algorithm and Implementation}
  \begin{itemize}
    \item \textbf{Basic idea}: find the longest suffix match in the given string (that is not a suffix itself)
  \end{itemize}

  \vspace{\fill}

  How we coded it:
  \begin{itemize}
    \item \textbf{Match pattern end indexes}: stored in a set
    \item \textbf{Scan set} to find longer matches until none is found
  \end{itemize}
  \vspace{\fill}

    Always better with an example...

\end{frame}

\begin{frame}{Example}\centering
  \input{code/simplified.tex}

  \only<1>{$$\text{ Set}=\{ 1,6,7,13,14,18,19,22,23,27 \}$$}
  
  \only<2>{$$\text{ Set}=\{ 7,14,19,23 \}$$}
  
  \only<3>{$$\text{ Set}=\{ 7,19,23 \}$$}
  
  \only<4>{$$\text{ Set}=\{ 7,19 \}$$}
\end{frame}

\tptsection{Universal predictor}

\begin{frame}{Algorithm and Implementation}
   \begin{itemize}
    \item \textbf{Basic idea}: uses previous idea to calculate maximal length $L$ of mathing pattern suffix. Then takes the set of matches of length $\alpha \cdot L$ and returns the character following those sequences with higher frequency.
  \end{itemize}

  \vspace{\fill}

  How we coded it:
  \begin{itemize}
    \item \textbf{Match pattern end indexes}: stored in a list of sets. Each set corresponds to one length
    \item \textbf{Scan last set in the list} to find longer matches until none is found
    \item Uses \textbf{set with matches of length $\alpha \cdot L$} to determine which following character is the most frequent
  \end{itemize}
  \vspace{\fill}

    Always better with an example...

\end{frame}

\begin{frame}{Example}\centering
  \input{code/complete.tex}

  \only<3>$$\text{ Set}=\{ 7,19,23 \}$$}
\end{frame}

\tptsection{Variation proposal}

\begin{frame}{Reading substrings}\centering
  \input{code/learning.tex}
\end{frame}

\tptsection{Experimentation}

\begin{frame}{Parameters influence}
\end{frame}

\begin{frame}{Distributed design}
\end{frame}

\tptsection{Comparison}

\end{document}
